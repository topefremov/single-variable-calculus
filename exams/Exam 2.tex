\documentclass{article}
\usepackage{graphicx} % Required for inserting images
\usepackage{tikz}
\usepackage{svg}
\usepackage{amsmath}
\usepackage{tcolorbox}

\title{Exam 2}
\author{Aleksandr Efremov}
\date{September 2023}

\begin{document}

\maketitle

\section{Problem 1}
\begin{tcolorbox}
Estimate the following to two decimal places (show work).
\end{tcolorbox}
\begin{itemize}
    \item[a.] \( \sin{(\pi + 1/100)} \) 
    \[ \sin{(\pi + 1/100)} \approx \frac{cos{\pi}}{100} + \sin{\pi} = -\frac{1}{100} + 0 = -0.01 \]
    \item[b.] \( \sqrt{101} \)
    \[ \sqrt{101} \approx \frac{1}{2\sqrt{100}}  \cdot 1 + \sqrt{100} = \frac{201}{20} = 10.05 \]
\end{itemize}
\section{Problem 2}
\begin{tcolorbox}
    Sketch the graph of \( y = \frac{4}{x} + x + 1 \) on \( -\infty < x < \infty \) and label all critical points and infection points with their coordinates on the graph along with the letter "C" or "I"
\end{tcolorbox}
Skipped.
\section{Problem 3}
\begin{tcolorbox}
     An architect plans to build a triangular enclosure with a fence on two sides and a wall on the third side. Each of the fence segments has fixed Length $L$. What is the length $x$ of the third side if the region enclosed has the largest possible area? Show work and include an argument to show that your answer really gives the maximum area. 
\end{tcolorbox}
\begin{figure}[htp!]
    \centering
    \includesvg{exam2-1.svg}
    \label{fig:fig1}
\end{figure}
The area $A$ is given by the formula
\[ A = \frac{x}{2}\sqrt{L^2 - \frac{x^2}{4}} = \frac{1}{4}\sqrt{4L^2x^2-x^4} \]

We consider only $x \geq 0$. The constraint is that $L^2 - \frac{x^2}{4} \geq 0 \Rightarrow x \leq 2L$. So if $x = 0$ or $x = 2L$, then $A = 0$. So the maximum area must be in between at critical point.
\[ \frac{d}{dx} \left( \frac{1}{4}\sqrt{4L^2x^2-x^4} \right) = 0 \]
\[ \frac{2L^2x - x^3}{\sqrt{L^2x^2-x^4}} = 0 \]
\[ x = \sqrt{2}L \]

\section{Problem 4}
\begin{tcolorbox}
     A rocket has launched straight up, and its altitude is $h = 10t^2$ feet after $t$ seconds. You are on the ground $1000$ feet from the launch site. The line of sight from you to the rocket makes an angle $\theta$ with the horizontal. By how many Radians per second is $\theta$ changing ten seconds after the launch? 
\end{tcolorbox}
\newpage
\begin{figure}[htp!]
    \centering
    \includesvg{exam2-2.svg}
    \label{fig:fig2}
\end{figure}
We need to find $\frac{d\theta}{dt}$ where $t$ is measured in seconds. We are given that $h = 10t^2$, therefore $\frac{dh}{dt}$ after $10$ seconds is $200$. To find $\frac{d\theta}{dt}$ we implicitly differentiate the equation $\tan{\theta} = \frac{h}{1000}$ with respect to $t$.
\[ \frac{d}{dt} \tan{\theta} = \frac{d}{dt} \frac{h}{1000} \]
\[ \sec^2{\theta}\frac{d\theta}{dt} = \frac{dh/dt}{1000} \]
\par If $t = 10$, then $h = 1000$ and $\sec^2{\theta} = \tan^2{\theta} + 1 = \frac{1000}{1000} + 1 = 2$. Then,
\[ 2\frac{d\theta}{dt} = \frac{200}{1000} \Rightarrow \frac{d\theta}{dt} = \frac{1}{10} \]

\section{Problem 5}
\begin{tcolorbox}
    a. Evaluate the following indefinite integrals.
\end{tcolorbox}
\begin{itemize}
    \item[i.] $\int{\cos(3x)dx}$
    \[ u = 3x; \quad du = 3dx \quad \Rightarrow \quad \frac{1}{3}\int{\cos{(u)}du} = \frac{1}{3}\sin{3x} + C \]
    \item[ii.] $\int{xe^{x^2}dx}$
    \[ u = x^2; \quad du = 2xdx \quad \Rightarrow \quad \frac{1}{2}\int{e^udu} = \frac{1}{2}e^{x^2} + C \]
\end{itemize}
\begin{tcolorbox}
    b. Find $y(x)$ such that $y' = \frac{1}{y^3}$ and $y(0) = 1$.
\end{tcolorbox}
\[ \frac{dy}{dx} = \frac{1}{y^3} \]
\[ y^3dy = dx \]

\par Taking the integral of both sides
\[ \int{y^3dy} = \int{dx} \]
\[ \frac{y^4}{4} + C = x + C \]
\[ y = (4x + C)^{1/4} \]
\[ y(0) = 1 \Rightarrow (4 \cdot 0 + C)^{1/4} = 1 \Rightarrow C = 1  \]
\[ y = (4x + 1)^{1/4} \]

\section{Problem 6}
\begin{tcolorbox}
    Suppose that $f'(x) =  e^{(x^2)}$, and $f(0)=10$. One can conclude from the mean value theorem that
    \[ A < f(1) < B \]
    for which numbers A and B? 
\end{tcolorbox}
The mean value theorem says
\[ \frac{f(1) - f(0)}{1 - 0} = f'(c), \quad \text{for some $c$ s.t. $0 < c < 1$} \]
\par Then $f(1) = e^{(c^2)} + 10$. If $0 < c < 1$, then $1 < f'(c) < e$. Hence, 
\[ 11 < f(1) < 10 + e \]


\end{document}
