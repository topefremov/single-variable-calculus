\documentclass{article}
\usepackage{graphicx} % Required for inserting images
\usepackage{tikz}
\usepackage{svg}
\usepackage{amsmath}
\usepackage{tcolorbox}

\title{Problem Set 4}
\author{Aleksandr Efremov}
\date{September 2023}

\begin{document}

\maketitle

\section{Part I}
\subsection{Lecture 14}

\subsubsection{2G-1b}
\[ f(x) = \ln{x} \quad \text{on } [1, 2] \]
\par By MVT,
\[ f'(c) = \frac{\ln{2} - \ln{1}}{2 - 1} = \ln{2} \]
\[ f'(x) = \frac{1}{x} \Rightarrow \frac{1}{c} = \ln{2} \Rightarrow c = \frac{1}{\ln{2}} \]

\subsubsection{2G-2b}
\[ \sqrt{1 + x} < 1 + x/2 \quad \text{if } x > 0 \]

By MVT,
\[ f(x) = f(0) + f'(c)x \quad 0 < c < x\]
\[ f(x) = 1 + \frac{x}{2\sqrt{1 + c}} < 1 + \frac{x}{2}; \quad \text{because } x > 0 \]

\subsubsection{2G-5}
\begin{tcolorbox}
    a) Suppose $f''(x)$ exists on an interval $I$ and $f(x)$ has a zero at three distinct points $a < b< c$ on $I$. Show there is a point $p$ on $[a, c]$ where $f''(p) = 0$.
\end{tcolorbox}
\par By MVT,
\[ \frac{f(b)-f(a)}{b-a} = 0 \Rightarrow f'(c_1) = 0 \quad a < c_1 < b \]
\[ \frac{f(c)-f(b)}{c-b} = 0 \Rightarrow f'(c_2) = 0 \quad b < c_2 < c \]
\par Again by MVT,
\[ \frac{f'(c_2)-f'(c_1)}{c_2 - c_1} = 0 \Rightarrow f''(p) = 0 \quad c_1 < p < c_2 \]
\par Since $p$ is between $c_1$ and $c_2$, it's also between $a$ and $c$.

\subsubsection{2G-6}
\begin{tcolorbox}
    Using the form (2) of the Mean-value Theorem, prove that on an interval $[a,b]$,
    \[  f'(x) > 0 \quad \Rightarrow \quad f(x) \text{ increasing;} \]
\end{tcolorbox}
By MVT,
\[ f(b) - f(a) = f'(c)(b-a) \]
Function is increasing on interval $[a,b]$ if $a < b$ and $f(a) < f(b)$. Assume that $f'(x) > 0$. We are given that $a < b$. Then,
\[ f'(c)(b-a) > 0 \Rightarrow f(b) - f(a) > 0 \Rightarrow f(b) > f(a) \]
Since $a < b$ and $f(a) < f(b)$, then $f$ is increasing. 
\begin{tcolorbox}
    \[  f'(x) = 0 \quad \Rightarrow \quad f(x) \text{ constant;} \]
\end{tcolorbox}
Function is constant on interval $[a,b]$ if $a < b$ and $f(a) = f(b)$. Assume that $f'(x) = 0$. We are given that $a < b$. Then,
\[ f'(c)(b-a) = 0 \Rightarrow f(b) - f(a) = 0 \Rightarrow f(b) = f(a) \]
Since $a < b$ and $f(a) = f(b)$, then $f$ is constant. 

\subsection{Lecture 15}

\subsubsection{3A-1}
\begin{tcolorbox}
    Compute the differentials $df(x)$ of the following functions.
    \begin{itemize}
        \item[d)] $d(e^{3x} \sin{x})$
        \item[e)] Express $dy$ in terms of $x$ and $dx$ if $\sqrt{x} + \sqrt{y} = 1$
    \end{itemize}
\end{tcolorbox}

\begin{itemize}
    \item[d)] 
    \[ d(e^{3x} \sin{x}) = (3e^{3x}\sin{x} + e^{3x}\cos{x})dx\]
    \item[e)]
    \[ y = (1 - \sqrt{x})^2 \Rightarrow dy = \left( \frac{\sqrt{x}-1}{\sqrt{x}} \right)dx = \left( 1 - \frac{1}{\sqrt{x}} \right) dx \]
\end{itemize}

\subsubsection{3A-2}
\begin{tcolorbox}
    Compute the following indefinite integrals
    \begin{itemize}
        \item[a)] $\int{(2x^4+3x^2+x+8)dx}$
        \item[c)] $\int{\sqrt{8+9x}dx}$
        \item[e)] $\int{\frac{x}{\sqrt{8-2x^2}}dx}$
        \item[g)] $\int{7x^4e^{x^5}dx}$
        \item[i)] $\int{\frac{dx}{3x+2}}$
        \item[k)] $\int{\frac{x}{x+5}dx}$ (Write $\frac{x}{x+5} = 1 + \dots$)
    \end{itemize}
\end{tcolorbox}
\begin{itemize}
    
    \item[a)]
    \[ \int{(2x^4+3x^2+x+8)dx} = \frac{2}{5}x^5 + x^3 + \frac{1}{2}x^2 + 8x + C \]
    
    \item[c)]
    \[ \int{\sqrt{8+9x}dx} \]
    \[ u = 8 + 9x; \quad du = 9dx \]
    \[ \frac{1}{9}\int{\sqrt{u}du} = \frac{2}{27}u^{3/2} + C = \frac{2}{27}(8+9x)^{3/2} + C \]

    \item[e)]
    \[ \int{\frac{x}{\sqrt{8-2x^2}}dx} \]
    \[ u = 8-2x^2; \quad du = -4xdx \]
    \[ -\frac{1}{4}\int{\frac{du}{\sqrt{u}}} = -\frac{\sqrt{u}}{2} + C = -\frac{\sqrt{8-2x^2}}{2} + C \]

    \item[g)] 
    \[ \int{7x^4e^{x^5}dx} \]
    \[ u = x^5; \quad du = 5x^4dx \]
    \[ \frac{7}{5}\int{e^udu} = \frac{7}{5}e^u + C = \frac{7}{5}e^{x^5} + C \]

    \item[i)]
    \[ \int{\frac{dx}{3x+2}} \]
    \[  u = 3x+2; \quad du = 3dx \]
    \[ \frac{1}{3}\int{\frac{du}{u}} = \frac{1}{3} \ln{|u|} + C = \frac{1}{3} \ln{|3x+2|} + C \]

    \item[k)] 
    \[ \int{\frac{x}{x+5}dx} = \int{\left( 1 - \frac{5}{x+5} \right)dx} = x - 5\ln{|x+5|} + C \]
\end{itemize}

\subsubsection{3A-3}
\begin{tcolorbox}
    Compute the following indefinite integrals
    \begin{itemize}
        \item[a)] $\int{\sin(5x)dx}$
        \item[c)] $\int{\cos^2{x}\sin{x}dx}$
        \item[e)] $\int{\sec^2{(x/5)}dx}$
        \item[g)] $\int{\sec^9{x}\tan{x}dx}$
    \end{itemize}
\end{tcolorbox}
\begin{itemize}
    \item[a)]
    \[ \int{\sin(5x)dx} \]
    \[ u = 5x; \quad du = 5dx \]
    \[ \frac{1}{5}\int{\sin(u)du} = -\frac{\cos(5x)}{5} + C \]

    \item[c)]
    \[ \int{\cos^2{x}\sin{x}dx} \]
    \[ u = \cos{x}; \quad du = -\sin{x}dx \]
    \[ -\int{u^2du} = -\frac{\cos^3{x}}{3} + C \]

    \item[e)]
    \[ \int{\sec^2{(x/5)}dx} \]
    \[ u = x/5; \quad du = (1/5)dx \]
    \[ \int{5\sec^2{(u)}du} = 5\tan{(x/5)} + C \]

    \item[g)] 
    \[ \int{\sec^9{x}\tan{x}dx} = \int{\frac{\sin{x}}{\cos^{10}{x}}dx} \]
    \[ u = \cos{x}; \quad du = -\sin{x}dx \]
    \[ \int{\frac{-du}{u^{10}}} = \frac{1}{9\cos^9{x}} + C \]
\end{itemize}

\subsection{Lecture 16. Differential equations; separating variables.}
\subsubsection{3F-1}
\begin{tcolorbox}
    Solve the following differential equations.
    \begin{itemize}
        \item[c)] $dy/dx = 3/\sqrt{y}$
        \item[d)] $dy/dx = xy^2$  
    \end{itemize}
\end{tcolorbox}
\begin{itemize}
    \item[c)]
    \[ dy/dx = 3/\sqrt{y} \]
    \[ \sqrt{y}dy = 3dx \Rightarrow \frac{2}{3}y^{3/2} = 3x + c \Rightarrow y = \left( \frac{9x}{2} + \frac{3c}{2} \right)^{2/3} \]
    \item[d)]
    \[ dy/dx = xy^2 \]
    \[ \frac{dy}{y^2} = xdx \Rightarrow -\frac{1}{y} = \frac{x^2}{2} + c \Rightarrow y = \frac{-1}{\frac{x^2}{2} + c} \]
\end{itemize}

\subsubsection{3F-2}
\begin{tcolorbox}
    Solve each differential equation with the given initial condition, and evaluate the solution at the given value of $x$: 
    \begin{itemize}
        \item[a)] $dy/dx = 4xy, \quad y(1) = 3. \quad \text{Find } y(3).$
        \item[e)] $dy/dx = e^y, \quad y(3) = 0. \quad \text{Find } y(0). \\
        \text{For which values of $x$ is the solution $y$ defined?}$
    \end{itemize}
\end{tcolorbox}
\begin{itemize}
    \item[a)]
    \[ dy/dx = 4xy \Rightarrow \frac{dy}{y} = 4xdx \Rightarrow \ln{|y|} = 2x^2 + c \]
    \[ x = 1 \Rightarrow y = 3 \Rightarrow \ln{3} = 2 + c \Rightarrow c = \ln{3} - 2 \]
    \[ \ln{|y|} = 2x^2 + \ln{3} - 2 \]
    \[ x = 3 \Rightarrow \ln{|y|} = 16 + \ln{3} \Rightarrow y = 3e^{16} \]

    \item[e)] 
    \[ dy/dx = e^y \Rightarrow \frac{dy}{e^y} = dx \Rightarrow -e^{-y} = x + c \]
    \[ x = 3 \Rightarrow y = 0 \Rightarrow -e^{0} = 3 + c \Rightarrow c = -4 \]
    \[ -e^{-y} = x - 4 \]
    \[ x = 0 \Rightarrow -e^{-y} = -4 \Rightarrow e^y = 1/4 \Rightarrow y = \ln{(1/4)} \]
    The solution $y$ is defined for $x < 4$.
\end{itemize}

\subsubsection{3F-4}
\begin{tcolorbox}
    Newton's law of cooling says that the rate of change of temperature is proportional to the temperature difference. In symbols, if a body is at a temperature $T$ at time $t$ and the surrounding region is at a constant temperature $T_e$ ($e$ for external), then the rate of change of $T$ is given by
    \[ dT/dt = k(T_e - T) \]
    \par The constant $k > 0$ is a constant of proportionality that depends properties of the body like specific heat and surface area. \\ \\
    \par b) Find the formula for $T$ if the initial temperature at time $t = 0$ is $T_o$. 
\end{tcolorbox}
\[ \frac{dT}{T_e - T} = kdt \Rightarrow -\ln{|T_e-T|} = kt + c \]
\[ t = 0 \Rightarrow T = T_0 \Rightarrow c = -\ln{|T_e - T_0|} \]
\[ -\ln{|T_e-T|} = kt -\ln{|T_e - T_0|} \Rightarrow \frac{1}{T_e - T} = \frac{e^{kt}}{T_e - T_0} \]
\[ T = T_e - \frac{T_e - T_0}{e^{kt}} \]

\begin{tcolorbox}
    c) Show that $T \to T_e$ as $t \to \infty$
\end{tcolorbox}
Since $k > 0$, then  $e^{kt} \to \infty$ as $t \to \infty$. Hence, $\frac{T_e - T_0}{e^{kt}} = 0$ as $x \to \infty$.
\[ \lim_{t \to \infty}{\left[ T_e - \frac{T_e - T_0}{e^{kt}} \right]} = T_e \]

\begin{tcolorbox}
    d) Suppose that an ingot leaves the forge at a temperature of $680^{\circ}$ Celsius in a room at $40^{\circ}$ Celsius. It cools to $200^{\circ}$ in eight hours. How many hours does it take to cool from $680^{\circ}$ to $50^{\circ}$? (It is simplest to keep track of the temperature difference $T - T_e$, rather than $T$. The temperature difference undergoes exponential decay.)
\end{tcolorbox}

We are given that $T_0 = 680^{\circ}$, $T_e = 40^{\circ}$, and if $t = 8$ then $T = 200^{\circ}$
\[ 200 =  40 - \frac{40 - 680}{e^{8k}} \Rightarrow k = \frac{\ln{4}}{8}\]
\par We need to find $t_1$ such that $T(t_1) = 50^{\circ}$. 

\[ 50 = 40 - \frac{40 - 680}{e^{t\frac{\ln{4}}{8}}} \Rightarrow e^{t\frac{\ln{4}}{8}} = 64 \Rightarrow t = 8\frac{\ln{64}}{\ln{8}} = 24 \]

\subsubsection{3F-8}
\begin{tcolorbox}
    b) Find all plane curves in the first quadrant such that for every point $P$ on the curve, $P$ bisects the part of the tangent line at $P$ that lies in the first quadrant.
\end{tcolorbox}
If point $P$ bisects the part of the tangent line at $P$ in the first quadrant then the tangent line must intersect $x$-axis and $y$-axis in the first quadrant. The picture is below.
\begin{figure}[htp!]
    \centering
    \includesvg{3F-8.svg}
    \label{fig:fig1}
\end{figure}
\par The picture above implies that $\sin{\theta} = \frac{2y}{h}$ and $\cos{\theta} = \frac{2x}{h}$. Then $OB = 2x$ and $OA = 2y$ and the derivative of the tangent line is $\frac{dy}{dx} = -\frac{y}{x}$.
\[ \frac{dy}{y} = -\frac{dx}{x} \]
\[ \ln{|y|} = -\ln{|x|} + c \]
Since we care only about the first quadrant, $y$ and $x$ are positive.
\[ \ln{y} + \ln{x} = c \Rightarrow yx = e^c \Rightarrow y = \frac{a}{x} \quad {\text{where $a = e^c$ and $a > 0$}} \]

\subsection{Lecture 18. Definite integral; summation notation.}

\subsubsection{3B-2}
\begin{tcolorbox}
    Find a $\Sigma$ notation expression for
    \begin{itemize}
        \item[a)] $3 - 5 + 7 - 9 + 11 - 13$
        \item[b)] $1 + 1/4 + 1/9 + \dots + 1/n^2$ 
    \end{itemize}
\end{tcolorbox}
\begin{itemize}
    \item[a)]
    \[ \sum\limits_{i=0}^{5} ((-1)^i(3+2i)) \]
    \item[b)]
    \[ \sum\limits_{i=1}^{n}\frac{1}{i^2} \]
\end{itemize}

\subsubsection{3B-3}
\begin{tcolorbox}
    Write the upper, lower, left and right Riemann sums for the following integrals, using 4 equal subintervals:
    \begin{itemize}
        \item[b)] $\int_{-1}^{3}{x^2 \, dx}$
    \end{itemize}
\end{tcolorbox}
\begin{itemize}
    \item[b)]
    \[ \text{left sum: } -1^2 + 0^2 + 1^2 + 2^2 = 6 \]
    \[ \text{right sum: } 0^2 + 1^2 + 2^2 + 3^2 = 14 \]
    \[ \text{upper sum: } -1^2 + 1^2 + 2^2 + 3^2 = 15 \]
    \[ \text{lower sum: } 0^2 + 0^2 + 1^2 + 2^2 = 5 \]
\end{itemize}

\subsubsection{3B-4}
\begin{tcolorbox}
    Calculate the difference between the upper and lower Riemann sums for the following integrals with $n$ intervals
    \begin{itemize}
        \item[a)] $\int_{0}^b{x^2 \, dx}$
    \end{itemize}
    \par Does the difference tend to zero as $n$ tends to infinity? 
\end{tcolorbox}
\begin{itemize}
    \item[a)] 
    \[ \int_{0}^b{x^2 \, dx} \]
    The upper sum is $\frac{b^3}{n^3} \left( 1^2 + 2^2 + \dots +  n^2 \right)$. The lower sum is $\frac{b^3}{n^3} \left( 1^2 + 2^2 + \dots +  (n-1)^2 \right)$. So the difference is $\frac{b^3}{n}$, and it's zero as $n \to \infty$.
\end{itemize}

\subsubsection{3B-5}
\begin{tcolorbox}
    Evaluate the limit, by relating it to a Riemann sum.
    \[ \lim_{n \to \infty} = \frac{\sin{(b/n)} + \sin{(2b/n)} + \dots + \sin{((n-1)b/n)} + \sin{(nb/n)}}{n} \]
\end{tcolorbox}
We need to find $\int_{0}^b{\frac{\sin{x}}{b}dx}$.


\[ \frac{\sin{(b/n)} + \sin{(2b/n)} + \dots + \sin{((n-1)b/n)} + \sin{(nb/n)}}{n} = \frac{1}{n} \sum\limits_{k=1}^n \sin{(k\frac{b}{n})} \]
We use the formula of Problem 9 in Section 6.3 of the book to obtain
\[ \sum\limits_{k=1}^n \sin{(k\frac{b}{n})} = \frac{1}{n} \left[ \frac{\sin{(b/2)}\sin{\left[ \frac{b}{2n}(n+1) \right]}}{\sin{(b/2n)}}\right] \]

\[ \lim_{n \to \infty} \sin{\left[ \frac{b}{2n}(n+1) \right]} = \sin{(b/2)} \quad \text{see Section 6.5} \]

\[ \frac{1}{n} \cdot \frac{1}{\sin{(b/2n)}} = \frac{2}{b} \cdot \frac{b}{2n} \cdot \frac{1}{\sin{(b/2n)}} = \frac{2}{b} \quad \text{as $n \to \infty$ (see Section 6.5)} \]

\[ \lim_{n \to \infty} \frac{1}{n} \left[ \frac{\sin{(b/2)}\sin{\left[ \frac{b}{2n}(n+1) \right]}}{\sin{(b/2n)}}\right] = \frac{2 \sin{(b/2)}\sin{(b/2)}}{b} = \frac{1 - \cos{b}}{b} \]

\subsubsection{4J-1}
\begin{tcolorbox}
    Suppose it takes $k$ units of energy to lift a cubic meter of water one meter. About how much energy $E$ will it take to pump dry a circular hole one meter in diameter and $100$ meters deep that is filled with water? (Give reasoning.) 
\end{tcolorbox}
The circular hole is a cylinder of radius $1/2$ meter. If we divide the cylinder into $n$ equal cylinders of height $\Delta y$, then the volume of each cylinder is given by the formula $\frac{\pi}{4}\Delta y$. Now the energy it takes to lift the cylinder of height $\Delta y$ at some depth $y$ is equal to $ky\frac{\pi}{4}\Delta y$. Now we need to sum up all the energies it takes to lift all the cylinders.
\[ k\frac{\pi}{4} \sum \limits_{i = 0}^n y_i \Delta y \]
\par If $n \to \infty$, then we have an integral
\[ k\frac{\pi}{4} \int_{0}^{100} {y \, dy} \]
\section{Part II}

\subsection{Problem 1}
\begin{tcolorbox}
    a) Use the mean value property to show that if $f(0) = 0$ and $f'(x) \geq 0$, then $f(x) \geq 0$ for all $x \geq 0$.
\end{tcolorbox}
The MVT says
\[ \frac{f(b) - f(a)}{b - a} = f'(c) \quad a \le c \le b \]

\par So we have
\[ \frac{f(x) - f(0)}{x - 0} = f'(c) \Rightarrow f(x) = f'(c)x \]

Since $x \geq 0$ and $f'(c) \geq 0$, then $f(x) \geq 0$.

\begin{tcolorbox}
    b) Deduce from part (a) that $\ln{(1 + x)} \leq x$ for $x \geq 0$. Hint: Use $f(x) = x - \ln{(1 + x)}$.
\end{tcolorbox}

Let $f(x) = x - \ln{(1+x)}$, then $f(0) = 0$ and $f'(x) = 1 - \frac{1}{1+x} = \frac{x}{1+x}$. Since $x \geq 0$, then $f'(x) \geq 0$. Hence from part (a) we can conclude that $f(x) \geq 0$.
\[ x - \ln{(1+x)} \geq 0 \Rightarrow x \geq \ln{(1+x)} \]

\begin{tcolorbox}
    c) Use the same method as in (b) to show $\ln{(1 + x)} \geq x - x^2/2$ and $\ln{(1 + x)} \leq x - x^2/2 + x^3/3$ for $x \geq 0$.
\end{tcolorbox}

Let $f(x) = \ln{(1+x)} - x + x^2/2$, then $f(0) = 0$ and
\[ f'(x) = \frac{1}{1+x} - 1 + x = \frac{x^2}{1+x} \]

\par Since $x \geq 0$, then $f'(x) \geq 0$. Hence, from part (a) we can conclude that $f(x) \geq 0$.
\[ \ln{(1+x)} - x + x^2/2 \geq 0 \Rightarrow \ln{(1+x) \geq x - x^2/2} \]

\par Let $g(x) = x - x^2/2 + x^3/3 - \ln{(1+x)}$, then $f(0) = 0$ and
\[ f'(x) = 1 - x + x^2 - \frac{1}{1+x} = \frac{x^3}{1+x} \]

\par Since $x \geq 0$, then $f'(x) \geq 0$. Hence, from part (a) we can conclude that $f(x) \geq 0$.
\[ x - x^2/2 + x^3/3 - \ln{(1+x)} \geq 0 \Rightarrow x - x^2/2 + x^3/3 \geq \ln{(1+x)}  \]

\begin{tcolorbox}
    d) Find the pattern in (b) and (c) and make a general conjecture.
\end{tcolorbox}
The pattern is $x - x^2/2 + x^3/3 - x^4/4 + \dots + (-1)^{n-1}x^n/n$. So, the conclusion is
\[ \ln{(1+x)} \leq \sum \limits_{k = 1}^{n} (-1)^{k-1}x^k/k \quad \text{if $n$ is odd} \]
\[\ln{(1+x)} \geq \sum \limits_{k = 1}^{n} (-1)^{k-1}x^k/k \quad \text{if $n$ is even} \]

\begin{tcolorbox}
    e) Show that $\ln(1 + x) \leq x$ for $-1 < x \leq 0$. (Use the change of variable $u = -x$.)
\end{tcolorbox}
Let $u = -x$. Then,
\[ \ln{(1-u)} \leq -u \quad \text{for $0 \leq u < 1$} \]
\par Let $f(u) = -u - \ln{(1-u)}$, then $f(0) = 0$.
\[ f'(u) = -1 + \frac{1}{1-u} = \frac{u}{1-u} \geq 0 \quad \text{for $u \geq 0$} \].
Since $f(0) = 0$ and $f'(u) \geq 0$ for $u \geq 0$, then by (a) $f(u) \geq 0$ for $u \geq 0$.
\[ -u - \ln{(1-u)} \geq 0 \Rightarrow -u \geq \ln{(1-u)} \Rightarrow \ln{(1+x)} \leq x \quad \text{for $-1 < x \leq 0$}  \]

\subsection{Problem 2}
\begin{tcolorbox}
    Show that both of the following integrals are correct:
    \[ \int{\frac{dx}{(1-x)^2} = \frac{1}{1-x}} \quad \text{and} \quad \int{\frac{dx}{(1-x)^2} = \frac{x}{1-x}} \]
    \par Explain.
\end{tcolorbox}
\[ \frac{d}{dx}\left(\frac{1}{1-x}\right) = \frac{1}{(1-x)^2} \quad \quad \frac{d}{dx}\left(\frac{x}{1-x}\right) = \frac{1}{(1-x)^2} \]

\par The integral of $f(x)$ is a family of functions $F(x) + c$. The integrals are correct, because $\frac{1}{1-x} = \frac{x}{1-x} + 1$.

\begin{tcolorbox}
    Show that both of the following integrals are correct, and explain.
    \[ \int{\tan{x}\sec^2{x} \, dx} = (1/2)\tan^2{x}; \quad  \quad \int{\tan{x}\sec^2{x} \, dx} = (1/2)\sec^2{x} \]
\end{tcolorbox}
\[ \frac{d}{dx}(1/2)\tan^2{x} = \tan{x}\sec^2{x} \quad \quad \frac{d}{dx}(1/2)\sec^2{x} = \tan{x}\sec^2{x} \]
\par The integral of $f(x)$ is a family of functions $F(x) + c$. The integrals are correct, because $(1/2)\tan^2{x} = (1/2)\sec^2{x} - 1/2$.


\subsection{Problem 3}
\begin{tcolorbox}
    A motorboat moving in still water is resisted by the water with a force proportional to its velocity $v$. Show that the velocity $t$ seconds after the power is shut off is given by the formula $v = v_0e^{-ct}$, where $c$ is a positive constant and $v_0$ is the velocity at the moment the power is shut off. Also, if $s$ is the distance the boat coasts in time $t$, find $s$ as a function of $t$ and sketch the graph of this function. Hint: Use Newton's second law of motion.
\end{tcolorbox}
Since the resistive force is proportional to the velocity of the motorboat, then we have an equation
\[ \frac{dv}{dt}m = -kv \quad \text{where $m$ is a constant mass and $k$ is some positive constant} \]
\[ \frac{dv}{v} = -(k/m) \, dt \]
\[ \ln{v} = -(k/m)t + c_1 \]
\[ v = e^{-(k/m)t} \cdot e^{c_1} \]
\par If $t = 0$, then $v = e^{c_1} = v_0$. Let $c = k/m$, then $v = v_0e^{-ct}$. \\
\par If $s$ is the distance the boat coasts in time $t$, then $\frac{ds}{dt} = v$.
\[ \frac{ds}{dt} = v_0e^{-ct} \]
\[ ds = v_0e^{-ct}dt \]
\[ s = -\frac{v_0}{c}e^{-ct}+k \]

\par Assume that $s(0) = 0$, then $k = \frac{v_0}{c}$.
\[ s = \frac{v_0}{c}\left( 1 - e^{-ct} \right) \]

\subsection{Problem 4}
\begin{tcolorbox}
    Air pressure satisfies the differential equation $dp/dh = -(.13)p$, where $h$ is the altitude from sea level measured in kilometers. \\
    \par a) At sea level the pressure is $1$ $kg/cm^2$. Solve the equation and find the pressure at the top of Mt. Everest (10 $km$). 
\end{tcolorbox}
\[ \frac{dp}{dh} = -(0.13)p \]
\[ \frac{dp}{p} = -(0.13)dh \]
\[ \ln{p} = -(0.13)h + C \]
\[ p = e^{-(0.13)h} \cdot e^C \]

\par At sea level the pressure is $1$ $kg/cm^2$. So, $p(0) = 1$.
\[ e^C = 1 \Rightarrow C = 0 \]
\[ p = e^{-(0.13)h}\]

\par The pressure at the top of Mt. Everest is $p(10) = e^{-1.3} \approx 0.27$ $kg/cm^2$.

\begin{tcolorbox}
    b) Find the difference in pressure between the top and bottom of the Green Building. (Pretend it's $100$ meters tall starting at sea level.) Compute the numerical value using a calculator. Then use instead the linear approximation to $e^x$ near $x = 0$ to estimate the percentage drop in pressure from the bottom to the top of the Green Building. 
\end{tcolorbox}
The difference in pressure is $\Delta p = p(0.1) - p(0) = e^{-0.013}-1 \approx -0.01291586498$ $kg/cm^2$.
The linear approximation to $e^x$ near $x = 0$ is
\[ e^x \approx 1 + x \]
\[ e^x - 1 \approx x \]

\par If $x = -0.013$, then $e^{-0.013} - 1 \approx -0.013$. The percentage drop in pressure from the bottom to the top of the Green Building is $-0.013 \times 100 = -1.3\%$.

\begin{tcolorbox}
    c) Use the linear approximation $\Delta p \approx p'(0)\Delta h$ and compute $p'(0)$ directly from the differential equation to find the drop in pressure from the bottom to top of the Green Building. Notice that this gives an answer without even knowing the solution to the differential equation. Compare with the approximation in part (b). What does the linear approximation $p'(0)\Delta h$ give for the pressure at the top of Mt. Everest? 
\end{tcolorbox}
The differential equation is 
\[ \frac{dp}{dh} = -(0.13)p \Rightarrow p'(0) = -(0.13)p(0) = -0.13 \]
If $\Delta h = 0.1$ (difference between the top and the bottom of Green Building), then
\[ \Delta p \approx -0.13 \cdot 0.1 = -0.013 \]
This approximation is equal to the approximation in part (b).
Now use this approximation to compute the pressure at the top of Mt. Everest. If $\Delta h = 10$, then
\[ \Delta y = -0.13 \cdot 10 = -1.3 \]
\[ p(10) = -1.3 + p(0) = -1.3 + 1 = -0.3  \]
This approximation makes no sense because $h \not\approx 0$.

\subsection{Problem 5}
\begin{tcolorbox}
    Calculate $\int_{0}^{1}{e^x \, dx}$ using lower Riemann sums. (You will need to sum a geometric series to get a usable formula for the Riemann sum. To take the limit of Riemann sums, you will need to evaluate $\lim_{n \to \infty} n(e^{1/n} - 1)$, which can be done using the standard linear approximation to the exponential function.)
\end{tcolorbox}
\[ S_n = e^0\frac{1}{n} + e^{1/n}\frac{1}{n} + \dots + e^{n-1/n}\frac{1}{n} = \frac{1}{n}\sum \limits_{k = 0}^{n-1} {e^{k/n}} = \frac{1 - e}{n(1 - e^{1/n})} \]

\[ \int_{0}^{1}{e^x \, dx} = \lim_{n \to \infty} \frac{1 - e}{n(1 - e^{1/n})}\]
\par If $n \to \infty$, then $1/n \to 0$. So we can use standard linear approximation to find $e^{1/n}$.
\[ e^x \approx 1 + x \Rightarrow e^{1/n} \approx 1 + 1/n \]
\[ \lim_{n \to \infty} n(1 - e^{1/n}) \approx  \lim_{n \to \infty} n(1 - (1 + 1/n)) = -1 \]
\[ \lim_{n \to \infty} \frac{1 - e}{n(1 - e^{1/n})} = \frac{1-e}{-1} = e - 1 \]

\subsection{Problem 6. More about the hypocycloid.}
\begin{tcolorbox}
    We use differential equations to find the curve with the property that the portion of its tangent line in the first quadrant has fixed length.
    \par a) Suppose that a line through the point $(x_0, y_0)$ has slope $m_0$ and that the point is in the first quadrant. Let $L$ denote the length of the portion of the line in the first quadrant. Calculate $L^2$ in terms of $x_0$, $y_0$ and $m_0$. (Do not expand or simplify.)
\end{tcolorbox}
The $x$-axis, $y$-axis and the line form a right triangle with hypotenuse $L$. The line equation is
\[ y - y_0 = m_0(x-x_0) \]
If $x = 0$, then $y = y_0 - m_0x_0$. If $y = 0$, then $x = x_0 - y_0/m_0$.
\[ L^2 = (y_0 - m_0x_0)^2 + (x_0 - y_0/m_0)^2 \]

\begin{tcolorbox}
    b) Suppose that $y = f(x)$ is a graph on $0 \leq x \leq L$ satisfying $f(0) = L$ and $f(L) = 0$ and such that the portion of each tangent line to the graph in the first quadrant has the same length $L$. Find the differential equation that $f$ satisfies. Express it in terms of $L$, $x$, $y$ and $y' = dy/dx$. (Hints: This requires only thought, not computation. Note that $y = f(x)$, $y' = f'(x)$. Don’t take square roots, the expression using $L^2$ is much easier to use. Don’t expand or simplify; that would make things harder in the next step.)
\end{tcolorbox}
The portion of he tangent line at any point $x$ on the interval $[0, L]$ has length $L$. It means that $L^2 = X^2 +Y^2$ for some $X$ and $Y$ which are in the first quadrant. We can relate this point $x$ to part (a) of the problem and say that $x_0 = x$, $y_0 = y$ and $m_0 = y'$. Then,
\[ L^2 = (y - y'x)^2 + (x - y/y')^2 \]

\begin{tcolorbox}
    c) Differentiate the equation in part (b) with respect to $x$. Simplify and write in the form
    \[ \text{(something)}(xy' - y)y'' = 0 \]
    (This starts out looking horrendous, but simplifies considerably.)
\end{tcolorbox}
\[ \frac{d}{dx}L^2 = \frac{d}{dx} \left[ (y - y'x)^2 + (x - y/y')^2 \right] \]
\[ 0 = \frac{d}{dx}(y - y'x)^2 + \frac{d}{dx}(x - y/y')^2 \]
\[ 0 = 2(y-y'x)(y' - y''x - y') + 2\frac{(xy'-y)}{y'}\frac{y''y}{(y')^2} \]
\[ 0 = -2x(y - y'x)y'' + \frac{2y}{(y')^3}(xy'-y)y'' \]
\[ 2\left(x + \frac{y}{(y')^3}\right)(xy'-y)y'' = 0 \]

\begin{tcolorbox}
    d) Show that one solution to the equation in part (c) is $x^{2/3} + y^{2/3} = L^{2/3}$. What about two other possibilities, namely, those solving $y'' = 0$ and $xy' - y = 0$?
\end{tcolorbox}
The solution to the equation in part (c) is some function $y = f(x)$ that satisfies the equation:
\[ x + \frac{y}{(y')^3} = 0 \Rightarrow y^{-1/3} \, dy = -x^{-1/3} \, dx \]
\[ 3/2y^{2/3} = -3/2x^{2/3} + C \]
\[ 3/2x^{2/3} + 3/2y^{2/3} =  C \]
\par Also note from part (b) that $f(0) = L$, and $f(L) = 0$. In either case we have:
\[ C = 3/2L^{2/3} \Rightarrow x^{2/3} + y^{2/3} = L^{2/3} \]
\par If $y'' = 0$, then function is a line having form $y = ax + b$.
\[ x = 0;\,\,y = L \quad \Rightarrow \quad   b = L\]
\[ x = L;\,\,y = 0 \quad \Rightarrow \quad  a = -1\]
\par So the solution is $y = L - x$.

\par If $xy' - y = 0$, then $y' = y/x$.
\[ \frac{dy}{dx} = \frac{y}{x} \Rightarrow \frac{dy}{y} = \frac{dx}{x} \]
\[ \ln{y} = \ln{x} + c \Rightarrow y = Ax \quad \text{(where $A = e^c > 0$ )} \]
\[ x = 0; \,\, y = L \quad \Rightarrow \quad L = 0 \]
\[ x = L; \,\, y = 0 \quad \Rightarrow \quad  AL = 0 \quad \Rightarrow \quad L = 0 \quad \text{(Since $A > 0$)}\]
\par So the solution is a point $(0, 0)$. 

\end{document}
